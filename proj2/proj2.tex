\documentclass[a4paper,11pt,twocolumn]{article}

\usepackage[czech]{babel}
\usepackage[IL2]{fontenc}
\usepackage{amsthm}
\usepackage{amsfonts}
\usepackage{amsmath}
\def\Vdash{\mathop{\vdash}}
\usepackage{times}
\usepackage[utf8]{inputenc}
\usepackage[left=1.5cm,text={18cm, 25cm},top=2.5cm]{geometry}

\newtheorem{thm}{Theorem}
\newtheorem{definice}[thm]{Definice}
\newtheorem{veta}{Věta}

\begin{document}
\begin{titlepage}
\begin{center}

{\Huge \textsc{Fakulta informačních technologií\\
Vysoké učení technické v~Brně}}\\

\vspace{\stretch{0.382}}

{\LARGE
Typografie a publikování\,--\,2. projekt\\
Sazba dokumentů a matematických výrazů\\
}

\vspace{\stretch{0.618}}
\end{center}

{\LARGE 2020 \hfill Lukáš Javorský (xjavor20)}
\end{titlepage}

\section*{Úvod}
\pagenumbering{arabic}
V~této úloze si vyzkoušíme sazbu titulní strany, ma\-te\-ma\-tic\-kých vzorců, prostředí a~dalších textových struktur ob\-vyk\-lých pro technicky zaměřené texty (například rovnice~(\ref{rov2})
nebo Definice~\ref{def2} na straně~\pageref{def2}). Pro vytvoření těchto odkazů
používáme příkazy \verb|\label|, \verb|\ref| a \verb|\pageref|.

Na titulní straně je využito sázení nadpisu podle optického středu s~využitím zlatého řezu. Tento postup byl
probírán na přednášce. Dále je použito odřádkování se
zadanou relativní velikostí 0.4em a 0.3em

\section{Matematický text}
Nejprve se podíváme na sázení matematických symbolů
a~výrazů v~plynulém textu včetně sazby definic a~vět s~využitím balíku \verb|amsthm|. Rovněž použijeme poznámku pod
čarou s~použitím příkazu \verb|\footnote|. Někdy je vhodné
použít konstrukci \verb|${}$| nebo \verb|\mbox{}| která říká, že
(matematický) text nemá být zalomen. V~následující definici je nastavena mezera mezi jednotlivými položkami
\verb|\item| na 0.05em.

\begin{definice}
    \textnormal{Turingův stroj} (TS) je definován jako šestice \label{def1}
    tvaru $M = (Q, \Sigma, \Gamma, \delta, q_0, q_F )$, kde:
\end{definice}
\begin{itemize}
    \item $Q$ \textit{je konečná množina} vnitřních (řídicích) stavů,
    \item $\Sigma$ \textit{je konečná množina symbolů nazývaná} vstupní
        abeceda, $\Delta \notin \Sigma$,
    \item $\Gamma$ \textit{je konečná množina symbolů, $\Sigma \subset \Gamma$, $\Delta \in \Gamma$,
        nazývaná} pásková abeceda
    \item $\delta$ : $(Q\backslash\{q_F \})\times\Gamma\rightarrow Q\times(\Gamma\cup\{L, R\})$, \textit{kde} $L, R \notin \Gamma$, je \textit{parciální} přechodová funkce, \textit{a}
    \item $q_0 \in Q$ \textit{je} počáteční stav \textit{a} $q_f \in Q$ je koncový stav.
\end{itemize}

Symbol $\Delta$ značí tzv. \textit{blank} (prázdný symbol), který se
vyskytuje na místech pásky, která nebyla ještě použita.

\textit{Konfigurace pásky} se skládá z~nekonečného řetězce,
který reprezentuje obsah pásky a pozice hlavy na tomto
řetězci. Jedná se o~prvek množiny \{$\gamma \Delta^\omega\, \vert\, \gamma \in \Gamma^*\} \times \mathbb{N}$\footnote{Pro libovolnou abecedu $\Sigma$ je $\Sigma^\omega$ množina všech \textit{nekonečných} řetězců nad $\Sigma$, tj. nekonečných posloupností symbolů ze $\Sigma$.}.
\textit{Konfiguraci} pásky obvykle zapisujeme jako $\Delta xyz\underline{z} x \Delta$...
(podtržení značí pozici hlavy). \textit{Konfigurace stroje} je pak
dána stavem řízení a~konfigurací pásky. Formálně se jedná
o~prvek množiny $Q \times \{\gamma \Delta^\omega\, \vert\, \gamma \in \Gamma^*\} \times \mathbb{N}$.

\subsection{Podsekce obsahující větu a odkaz}
\begin{definice}
    \textnormal{ Řetězec $w$ nad abecedou $\Sigma$ je přijat TS} M \label{def2}
    jestliže M při aktivaci z~počáteční konfigurace pásky\linebreak\newpage 
    \noindent$\underline{\Delta}w \Delta$... a počátečního stavu $q_0$ zastaví přechodem do\linebreak
    koncového stavu $q_F$ , tj. $\displaystyle (q_0,\Delta \omega \Delta^\omega, 0) \Vdash_M^* abc (q_F, \gamma, n)$ pro nějaké $\gamma \in \Gamma^*$ a $n \in \mathbb{N}$

Množinu $L(M) = \{w \, \vert\, w$ je přijat TS $M\}  \subseteq \Sigma^*$ nazýváme jazyk přijímaný \textnormal{TS} $M$.
\end{definice}

Nyní si vyzkoušíme sazbu vět a~důkazů opět s~použitím
balíku \verb|amsthm|.

\begin{veta}
    Třída jazyků, které jsou přijímány TS, odpovídá rekurzivně vyčíslitelným jazykům
\end{veta}

\noindent\textit{Důkaz.}  V~důkaze vyjdeme z~Definice \ref{def1} a \ref{def2}.

\section{Rovnice}
Složitější matematické formulace sázíme mimo plynulý
text. Lze umístit několik výrazů na jeden řádek, ale pak je
třeba tyto vhodně oddělit, například příkazem \verb|\quad|.

$$\sqrt[3]{x^3_i} \quad \textnormal{kde}\ x_i\ \textnormal{je}\ i \textnormal{-té sudé číslo}\quad y^{2\cdot x_i}_i \neq y^{y^{y_i}_i}_i $$ 

V~rovnici (\ref{rov1}) jsou využity tři typy závorek s~různou explicitně definovanou velikostí.

\begin{eqnarray}
    x & = & \bigg\{ \Big( \big[a + b \big] * c  \Big) ^d \oplus 1 \bigg\} \label{rov1}\\ 
    y & = & \lim_{x\rightarrow\infty} \frac{ \sin^2x + \cos^2x}{\frac{1}{\log_{10} x}} \label{rov2} 
\end{eqnarray}

V~této větě vidíme, jak vypadá implicitní vysázení limity $\lim_{n\rightarrow\infty}f(n)$ v~normálním odstavci textu. Podobně je to i s~dalšími symboly jako $\sum_{i=1}^n 2^i$ či $\bigcap_{A \in B} A$. V~případě vzorců $\lim\limits_{x\rightarrow\infty} f(n)$ a $\sum\limits _{i=1}^n 2^i$ jsme si vynutili méně úspornou sazbu příkazem \verb|\limits|.

\section{Matice}
Pro sázení matic se velmi často používá prostředí \verb|array| a závorky (\verb|\left|, \verb|\right|).

$$
\left(
    \begin{array}{ccc}
        a + b & \widehat{\xi + \omega} & \hat{\pi}\\
        \vec{\textnormal{\textbf{a}}} & \overleftrightarrow{AC} & \beta  
    \end{array}
\right) = 1 \Longleftrightarrow \mathbb{Q} = \mathcal{R}
$$ 

\noindent Prostředí \verb|array| lze úspěšně využít i~jinde.

$$ 
\binom{n}{i} = 
\left\{
    \begin{array}{c l}
        0 & \text{pro } k < 0 \ \text{nebo } \ k > n \\
        \frac{n!}{k!(n - k)!} & \text{pro}\ 0 \leq k \leq n.
    \end{array}
\right.
$$
\end{document}